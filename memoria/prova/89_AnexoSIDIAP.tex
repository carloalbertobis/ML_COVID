% Options for packages loaded elsewhere
\PassOptionsToPackage{unicode}{hyperref}
\PassOptionsToPackage{hyphens}{url}
%
\documentclass[
]{article}
\usepackage{amsmath,amssymb}
\usepackage{lmodern}
\usepackage{iftex}
\ifPDFTeX
  \usepackage[T1]{fontenc}
  \usepackage[utf8]{inputenc}
  \usepackage{textcomp} % provide euro and other symbols
\else % if luatex or xetex
  \usepackage{unicode-math}
  \defaultfontfeatures{Scale=MatchLowercase}
  \defaultfontfeatures[\rmfamily]{Ligatures=TeX,Scale=1}
\fi
% Use upquote if available, for straight quotes in verbatim environments
\IfFileExists{upquote.sty}{\usepackage{upquote}}{}
\IfFileExists{microtype.sty}{% use microtype if available
  \usepackage[]{microtype}
  \UseMicrotypeSet[protrusion]{basicmath} % disable protrusion for tt fonts
}{}
\makeatletter
\@ifundefined{KOMAClassName}{% if non-KOMA class
  \IfFileExists{parskip.sty}{%
    \usepackage{parskip}
  }{% else
    \setlength{\parindent}{0pt}
    \setlength{\parskip}{6pt plus 2pt minus 1pt}}
}{% if KOMA class
  \KOMAoptions{parskip=half}}
\makeatother
\usepackage{xcolor}
\usepackage[margin=1in]{geometry}
\usepackage{graphicx}
\makeatletter
\def\maxwidth{\ifdim\Gin@nat@width>\linewidth\linewidth\else\Gin@nat@width\fi}
\def\maxheight{\ifdim\Gin@nat@height>\textheight\textheight\else\Gin@nat@height\fi}
\makeatother
% Scale images if necessary, so that they will not overflow the page
% margins by default, and it is still possible to overwrite the defaults
% using explicit options in \includegraphics[width, height, ...]{}
\setkeys{Gin}{width=\maxwidth,height=\maxheight,keepaspectratio}
% Set default figure placement to htbp
\makeatletter
\def\fps@figure{htbp}
\makeatother
\setlength{\emergencystretch}{3em} % prevent overfull lines
\providecommand{\tightlist}{%
  \setlength{\itemsep}{0pt}\setlength{\parskip}{0pt}}
\setcounter{secnumdepth}{-\maxdimen} % remove section numbering
\ifLuaTeX
  \usepackage{selnolig}  % disable illegal ligatures
\fi
\IfFileExists{bookmark.sty}{\usepackage{bookmark}}{\usepackage{hyperref}}
\IfFileExists{xurl.sty}{\usepackage{xurl}}{} % add URL line breaks if available
\urlstyle{same} % disable monospaced font for URLs
\hypersetup{
  hidelinks,
  pdfcreator={LaTeX via pandoc}}

\author{}
\date{\vspace{-2.5em}}

\begin{document}

\hypertarget{anexo---sidiap}{%
\section{Anexo - SIDIAP}\label{anexo---sidiap}}

Como tenía que ser (y como -- probablemente será). Este estudio se tenía
que desarrollar con la base de datos incluye todos los registros de las
personas incluidos en Institut Català de la Salut (ICS) desde la primera
fecha reportada de caso COVID-19 en la base de datos y el cierre de la
instancia (31 diciembre 2022). La idea de este trabajo es utilizar una
base de datos EHR para generar información útil y apoyar a la
investigación y en un futuro utilizar algoritmos útiles para la toma de
decisiones en el sector sanitario.

Las parciales novedades de este estudio, en comparación a otros estudios
que se han publicado hasta la fecha, es la inclusión de la población
vacunada dentro el modelo predictivo. Esto significa que la vacuna es
una variable independiente, y que podría ser estadísticamente
significativa a la hora de calcular la variable dependiente (el output),
en este caso, la severidad del COVID. Incluyendo la variable vacuna, se
podría predecir la severidad no solamente en la población no vacunada,
también entre la población vacunada que no es completamente exenta de
riesgos. De la misma manera se podría verificar si realmente la vacuna
ha sido eficaz en la prevención.

Otra novedad, es que para este estudio como explicado antes, se tenía
que utilizar la base de datos SIDIAP. SIDIAP se ha creado en el 2010, e
incluye registros desde el año 2006. Actualmente tiene los datos
pseumonizados del 80\% de la población catalana por un total de 5,7
millones de personas. La base de datos SIDIAP incluyen los eventos de
cada persona en caso de derivar a la atención primaria, urgencias y
hospitales. Además, se dispone de información de las personas sobre las
enfermedades, medicamentos, observaciones clínicas, variables
bioquímicas del análisis de la sangre y urina, variables físicas (BMI),
variables socioeconómicas, vacunas, el uso de alcohol y tabaco. De
consecuencia todos los sujetos presentes en la base de datos son
potencialmente sujetos del estudio. A diferencia de otros estudios
similares en los cuales hay un numero de observaciones (n) relativamente
reducido, es este caso, sería posible entrenar el modelo con un numero
de observaciones elevado.

El estudio permitía abarcar diferentes objetivos, y principalmente dos:

\begin{itemize}
\item
  Determinar las variables estadísticamente significativas que estén
  relacionada con la severidad Covid, o más en general determinar los
  factores de riesgo relacionados con la severidad COVID.
\item
  Valorar y comparar diferentes modelos de Machine Learning
\end{itemize}

Objetivos secundarios alcanzables.

\begin{itemize}
\item
  Estimar la efectividad de las vacunas sobre la severidad de la Covid.
\item
  Determinar niveles de riesgo con una, dos o tres vacunas, o si hay una
  diferencia estadísticamente significativa.
\item
  No limitar el estudio a la severidad del COVID, también estudiar los
  días de positividad, y lo factores que pueden influir sobre los
  factores de riesgo
\end{itemize}

\hypertarget{common-data-model}{%
\subsection{Common Data Model}\label{common-data-model}}

En principio la idea era organizar los datos de un CDM, los CDM son una
forma de organizar los datos EHR. Son protocolos de organización los
datos, hay protocolos diferentes y estos cambian en base a los grupos de
investigación. Por ejemplo, dos que se está utilizando para el estudio
de COVID son OMOP y CONCEPTION.

Los beneficios que puede aportar el utilizo de un protocolo de
organización de los datos. Si se armoniza la estructura de los datos en
un standard aceptado por la comunidad. Los datos de esta forma están
organizados según un esquema utilizado por diferentes investigadores de
diferentes centros. Hay la posibilidad de comparar diferentes bases de
datos, utilizando el mismo script por diferentes bases de datos se
permitiría una cierta escalabilidad en la investigación y se podría
ampliar más fácilmente el estudio a otras bases de datos. Además, sería
un mecanismo que se autoalimenta, permitiendo el desarrollo del script
mejorando los existentes y creando de nuevos.

\hypertarget{variables}{%
\subsection{Variables}\label{variables}}

El tamaño muestral influye negativamente en las prestaciones y los
tiempos computacionales. Pero esta tarea se tenía que llevar con un
server del mismo centro bastante potente para poder llevar a cabo los
cálculos de ML.

PersonID (pseudominizado) Fecha Nacimiento

Fecha muerte Sexo

Covid:

\begin{itemize}
\item
  Fecha Vacunación
\item
  Numero de Dosis
\item
  Marca vacuna
\item
  Fecha diagnostico contagio (RT-PCR o antígeno)
\end{itemize}

Variables Analíticas

\begin{itemize}
\item
  Variables Analíticas
\item
  Fecha del análisis
\end{itemize}

Variables Clínicas.

\begin{itemize}
\item
  Comorbilidades. Se identifican los pacientes través el código de
  diagnóstico de la enfermedad o través la medicación que la persona
  está asumiendo
\item
  Fecha de diagnostico
\end{itemize}

Estilo de vida.

\begin{itemize}
\item
  Alcoholismo
\item
  Tabaquismo
\item
  BMI
\item
  Fecha de los diagnósticos
\end{itemize}

\hypertarget{criterios-de-exclusiuxf3n-e-inclusiuxf3n}{%
\subsection{Criterios de exclusión e
inclusión}\label{criterios-de-exclusiuxf3n-e-inclusiuxf3n}}

Los criterios de inclusión son amplios y intentan recoger el mayor
número de personas.

Los únicos criterios de exclusión se pueden considerar:

\begin{itemize}
\item
  Periodo de estudio, primer caso reportado de positividad COVID, hasta
  el cierre de la base de datos (actualmente 31 diciembre 2022)
\item
  Los menores de edad a la fecha de inicio del estudio. La severidad de
  COVID-19 en los menores desde el 2020 a la fecha de cierre del estudio
  ha estado relativamente baja, con las variantes delta u ómicron los
  menores se han visto más afectados. Esto se deduce de la incidencia de
  hospitalización, relativamente baja en todo el periodo de estudio,
  también la incidencia de hospitalización en UCI y la incidencia de
  muerte has sido bajas para este grupo de estudio.
\end{itemize}

Otros criterios de exclusión de deben a datos faltantes:

\begin{itemize}
\item
  En caso de positividad no tener la variable dependiente (severidad
  covid) significaría que hay datos faltantes o no correctos
\item
  En caso de no tener las variables (predictoras) de las personas. En el
  caso das variables de clínicas, analíticas y de estilo de vida es
  necesario que tengan una fecha anterior a la facha de contraer COVID,
  y que sea suficientemente recientes para que tengan valor clínico. De
  esta forma se ha elegido como criterio de exclusión, que estas
  variables sean máximo seis meses anteriores al inicio del estudio para
  las personas que no se han contagiado en el periodo de estudio y
  máximo seis meses anteriores a la fecha registrada de primer contagio
  de la persona.
\end{itemize}

\pagebreak

\end{document}
